\label{DistractionExperiment}
% ---------------------------------------------------
\section{Introduction}
Laxity tests are used by physicians to assess the integrity of specific ligaments. These tests can involve the physician moving the joint to a specific position and manually applying loads to the joint. Different tests are used for specific ligaments, and the physician assess the amount of joint motion and restraint that is provided (or not provided) by the targeted ligament. In a research setting, similar laxity-style tests are conducted on cadaveric specimens with custom fixtures \citep{roth_joshua_d._limits_2015,walker_effect_2014,rachmat_-situ_2016}\texthl{CITATIONS}, or six degree-of-freedom robots \texthl{CITATIONS}. Previous work has used physical experiments to quantify the effect of injury \texthl{CITATION}, and different techniques for ligament reconstruction \texthl{CITATION}, joint replacement \texthl{CITATION},

Experimental data were collected for use in the inverse modeling scheme described in \autoref{InverseModeling}. The purpose of the distraction experiments is to collect specimen-specific kinematic and kinetic data under traditional laxity-style loads, and novel distraction laxity-style loads. The traditional laxity-style loads are meant to simulate clinical tests that a physician may use to assess the integrity of specific ligaments. For a given joint position, the 

This section describes the experimental protocol that was used to test specimens under traditional and distraction laxity-style tests.

\section{Methods}
\subsection{Initial Specimen Preparation and Testing}
Experimental tests were conducted on two knee specimens (\texthl{AGE, BMI}). The specimens were initially prepared following the OpenKnee(s) protocol \citep{erdemir_open_2016}. In short, whole leg specimens from femoral head to foot were initially dissected by removing the soft tissue proximal and distal to the knee, leaving soft tissues intact from 8 cm proximal to the joint line to 8 cm distal. To facilitate model generation, three registration markers were fixed to the femur and tibia (six markers total) approximately 8 cm proximal and distal to the joint line, respectively. Optoelectronic sensors were fixed to the femur and tibia, and the position and orientation of the sensors was measured with an optoelectronic camera system (Optotrak, Northern Digital Inc., Waterloo, Ontario, Canada). Osseous landmarks (medial and lateral femoral epicondyles, femoral head, medial and lateral tibial plateau, and the medial and lateral malleolus) and ten points around each registration marker were digitized. After digitization, the tibia and femur were cut approximately 19 cm proximal and distal to the joint line, respectively. The specimen was them MR imaged. Following MR imaging, the femur and tibia were potted into fixtures that are used to mount to the robot.

A six degree-of-freedom simVITRO\texttrademark robot (Cleveland Clinic, Cleveland, OH) was used to apply laxity style loading to the specimen. Following this testing, and orthopedic surgeon dissected the skin, muscle, patella, and menisci, leaving the ligament structures intact. \texthl{Note the osteoarthritis}. The laxity-style tests were performed again on the now cleaned specimen. These tests were performed to increase the amount of data that was collected. The details of these tests are not described here because they do not relate to the distraction testing.

\subsection{Distraction Testing}
The specimens were dissected again following the intact and cleaned testing. An orthopedic surgeon 

\section{Results}
